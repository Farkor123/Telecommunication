\documentclass[a4paper, portrait,11pt]{article}
\usepackage[verbose,a4paper,tmargin=3cm,bmargin=3cm,lmargin=2.5cm,rmargin=2.5cm]{geometry}
\usepackage[utf8]{inputenc}
\usepackage{polski}
\usepackage{amsmath}
\usepackage{amsfonts}
\usepackage{amssymb}
\usepackage{lastpage}
\usepackage{indentfirst}
\usepackage{verbatim}
\usepackage{graphicx}
\usepackage{fancyhdr}
\usepackage{listings}
\usepackage{color}
\frenchspacing
\pagestyle{fancyplain}
\fancyhf{}
\renewcommand{\headrulewidth}{0pt}
\renewcommand{\footrulewidth}{0.4pt}
\newcommand{\degree}{\ensuremath{^{\circ}}} 
\fancyfoot[L]{Telekomunikacja: Hubert Sosnowski i Adam Sadowski}
\fancyfoot[R]{\thepage\ / \pageref{LastPage}}

\begin{document}

\begin{titlepage}
\begin{center}
\begin{tabular}{rl}
\begin{tabular}{|r|}
\hline \\
\large{\underline{210323~~~~~~~~~~~~~~~~~~~~~~~~~~~~~~~~~~~~~~~~~~~} }\\
$^{numer\ indeksu}$\\
\large {\underline{Huebrt Sosnowski~~~~~~~~~~~~~~~~~~~~~~~~~~~~} }\\
$^{imię\ i\ nazwisko}$ \\\\ \hline
\end{tabular} 
&
\begin{tabular}{|r|}
\hline \\
\large{\underline{210310~~~~~~~~~~~~~~~~~~~~~~~~~~~~~~~~~~~~~~~~~~~} }\\
$^{numer\ indeksu}$\\
\large {\underline{Adam Sadowski~~~~~~~~~~~~~~~~~~~~~~~~~~~~~~~~} }\\
$^{imię\ i\ nazwisko}$ \\\\ \hline
\end{tabular} 

\end{tabular}
~\\~\\~\\ 
\end{center}
\begin{tabular}{ll}
\LARGE{\textbf{Data}}& \LARGE{2018-03-19}\\
\LARGE{\textbf{Kierunek}}& \LARGE{Informatyka}\\
\LARGE{\textbf{Rok akademicki}}& \LARGE{2017/18} \\
\LARGE{\textbf{Semestr}}& \LARGE{4} \\
\LARGE{\textbf{Grupa dziekańska}}& \LARGE{1} \\\\\\\\\\
\end{tabular}

\begin{center}
\textbf{\Huge{Laboratorium\\ telekomunikacji\\~\\Zadanie 1}} \\~\\
\end{center}

\end{titlepage}
\setcounter{page}{2}

\subsection*{Korekcja jednego błędu}
Macierz H do tworzenia sumy kontrolnej oraz korekcji jednego błędu informacji 8-bitowej 
powinna spełniać następujące warunki: żadna z jej kolumn nie powinna się powtarzać oraz nie powinna 
posiadać kolumny zerowej. Skoro wiemy, że szukana macierz jest $n x (n+m)$ wymiarową, gdzie 
n-liczba bitów parzystości i m-liczba bitów wiadomości, to najmniejsza macierz o takich wymiarach, 
spełniająca powyższe warunki będzie macierz $4 x 12$ wymiarową.\\
\\
Macierz spełniająca powyższe warunki została przez nas dobrana w empiryczny sposób i potwierdziliśmy 
jej działanie w programie.\\

$$H_1 = \left[\begin{array}{ccccccccccccc}
0 & 1 & 1 & 1 & 1 & 0 & 0 & 1 & \, & 1 & 0 & 0 & 0 \\
1 & 0 & 1 & 1 & 0 & 0 & 1 & 1 & \, & 0 & 1 & 0 & 0 \\
0 & 1 & 1 & 0 & 1 & 1 & 1 & 1 & \, & 0 & 0 & 1 & 0 \\
1 & 0 & 0 & 1 & 1 & 1 & 1 & 1 & \, & 0 & 0 & 0 & 1
\end{array}\right]$$
\\
Do wyliczenia sumy kontrolnej informacji użyliśmy wzoru:\\

$c_i = \displaystyle\sum_{j=0}^{7} a_j*H_{i,j}$, 
gdzie i-indeks bitu sumy kontrolnej {0-3}, a-informacja, H-macierz.\\
\\
